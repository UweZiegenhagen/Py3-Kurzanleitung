%!TEX TS-program = Arara
% arara: pdflatex: {shell: yes}
\documentclass[12pt,ngerman]{scrreprt}

\usepackage[utf8]{inputenc}
\usepackage[T1]{fontenc}
\usepackage{booktabs}
\usepackage{babel}
\usepackage{graphicx}
\usepackage{csquotes}
\usepackage{paralist}
\usepackage{microtype}
\usepackage{xcolor}
\usepackage{cmbright}
\setlength{\parindent}{0pt}
\setlength{\parskip}{6pt}

\usepackage{scrpage2}
\pagestyle{scrheadings}
\setheadsepline[\textwidth]{1pt}{}
 \ohead{\headmark}
\ofoot[\pagemark]{\pagemark}
\cfoot[]{}
\chead[]{}

\usepackage{listings}
\lstset{%
    float=hbp,%
    basicstyle=\ttfamily\small, %
    identifierstyle=\color{colIdentifier}, %
    keywordstyle=\color{colKeys}, %
    stringstyle=\color{colString}, %
    commentstyle=\color{colComments}, %
    columns=flexible, %
    tabsize=2, %
    frame=single, %
    extendedchars=true, %
    showspaces=false, %
    showstringspaces=false, %
    numbers=left, %
    numberstyle=\tiny, %
    breaklines=true, %
    backgroundcolor=\color{hellgelb}, %
    breakautoindent=true, %
    captionpos=b%
}

\lstset{literate=%
    {Ö}{{\"O}}1
    {Ä}{{\"A}}1
    {Ü}{{\"U}}1
    {ß}{{\ss}}1
    {ü}{{\"u}}1
    {ä}{{\"a}}1
    {ö}{{\"o}}1
    {~}{{\textasciitilde}}1
}

\usepackage{minted}

\usepackage{caption}
\captionsetup{font={normalsize}, textfont={sf}, labelfont={bf,sf}}

\renewcommand{\listingscaption}{{\sffamily Code}}


\usepackage{hyperref}
\hypersetup{
    bookmarks=true,                     % show bookmarks bar
    unicode=false,                      % non - Latin characters in Acrobat’s bookmarks
    pdftoolbar=true,                        % show Acrobat’s toolbar
    pdfmenubar=true,                        % show Acrobat’s menu
    pdffitwindow=false,                 % window fit to page when opened
    pdfstartview={FitH},                    % fits the width of the page to the window
    pdftitle={Python Kurzeinführung},                        % title
    pdfauthor={Uwe Ziegenhagen},                 % author
    pdfsubject={Python},                   % subject of the document
    pdfcreator={Uwe Ziegenhagen},                   % creator of the document
    pdfproducer={Producer},             % producer of the document
    pdfkeywords={Python,Introduction},   % list of keywords
    pdfnewwindow=true,                  % links in new window
    colorlinks=true,                        % false: boxed links; true: colored links
    linkcolor=blue,                          % color of internal links
    filecolor=black,                     % color of file links
    citecolor=black,                     % color of file links
    urlcolor=black                        % color of external links
}



\title{Eine Kurzeinführung in Python~3}
\subtitle{-- DRAFT --}
\author{Uwe Ziegenhagen}


\begin{document}
\maketitle

\tableofcontents

\listoffigures

\listoftables

\listoflistings

\clearpage

\definecolor{bg}{rgb}{0.95,0.95,0.95}

\begin{minted}[
               linenos,
               bgcolor=bg,
               numbersep=5pt,
               gobble=2,
               label={aaa},
               frame=lines,
               framesep=2mm]{python}
print('Hello World')
\end{minted}

\mint{python}|import this|


\inputminted[bgcolor=bg]{python}{Codes/hello_world.py}.

    \begin{listing}[H]
        \caption{11 This is below the code.}
        \inputminted{python}{Codes/hello_world.py}
        \label{lst:the-code}
    \end{listing}


\section*{Über dieses Dokument}

Mit diesem Dokument versuche ich, auf wenigen Seiten einen Überblick über Python~3 zu geben. Ich habe nicht das Ziel, alle Aspekte der Sprache umfassend zu behandeln, ich möchte vielmehr den Einstieg in die Programmierung mit Python ermöglichen und zeigen, was man alles mit Python und ausgewählten Python-Modulen machen kann. 

Aus Gründen der Einfachheit werden daher einzelne Aspekte etwas verkürzt dargestellt, die geneigten Leserinnen und Leser mögen es mir nachsehen. 

Fehler jedoch bitte ich zu melden, am besten per E-Mail an ziegenhagen@gmail.com


\vspace*{3em}Köln, den \today

\vfill Dieses Dokument entstand mit \LaTeX\ und der CM Bright Schrift.

\chapter{Was ist eigentlich Python und warum sollte ich es können?}

Python ist eine Programmiersprache, die Anfang der 1990er Jahre von Guido van Rossum am Centrum Wiskunde \& Informatica in Amsterdam in den Niederlanden entwickelt wurde. Python gilt als sehr gut lesbar und lernbar und eignet sich daher auch gut als erste Programmiersprache. 

Python ist dabei eine sogenannte interpretierte Sprache, was heißt, dass man keinen Compiler wie beispielsweise in C und C++ benötigt, um ein Programm laufen zu lassen. Python-Programme sind daher meist etwas langsamer als kompilierte Programme, im praktischen Alltag spielt dies jedoch nur selten eine Rolle.

Lange Zeit gab es zwei parallele Entwicklungsstränge von Python, Python~2 und Python~3. Da Python~2 jetzt Anfang 2020 offiziell den Status \enquote{deprecated}, also  \enquote{überholt}, hat, betrachten wir in diesem Dokument nur Python~3.

\chapter{Installation}

\begin{itemize}
	\item Unter Linux ist Python üblicherweise vorinstalliert, oft ist jedoch noch -- zumindest Stand Anfang 2020 -- Python~2.7 das Standard-Python, nicht Python~3. 
	\item Unter Windows und Mac OS X ist standardmäßig kein Python installiert, hier muss man also selbst eine Python-Distribution installieren.
\item Neben dem Standardpython, das man sich von \url{https://www.python.org/downloads} herunterladen kann, gibt es noch alternative Distributionen.
	\item Die bekannteste Distribution ist Anaconda Python, siehe \url{https://www.anaconda.com/distribution}
	\item Als Alternative zu Anaconda ist WinPython (\url{https://winpython.github.io/}) empfehlenswert.
\end{itemize}

\section{Installation von Python 3.8 unter Windows}


\section{Installation von Anaconda unter Windows}


\chapter{Python als Taschenrechner}

Man kann Python auf verschiedene Arten benutzen:

\begin{enumerate}
\item interaktiv wie einen Taschenrechner
\item im Batch-Modus, bei dem die Befehle alle in einer Datei stehen und dann in \enquote{einem Rutsch} (englisch \enquote{Batch}) abgearbeitet werden
\item über halb-interaktive Methoden wie Jupyter und iPython
\end{enumerate}

In diesem Kapitel betrachten wir kurz die interaktive Nutzung, in den weiteren Kapiteln werden wir dann näher auf die Batch-Nutzung/ halb-interaktive Nutzung eingehen.

\chapter{Funktionen}

\chapter{Anwendungen}

\end{document}